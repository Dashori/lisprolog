\section{Практические задания}

\subsection{Задание \textnumero1}

Сдано на отдельном листе.

\subsection{Задание \textnumero2}

Используя только функции CAR и CDR, написать выражения, возвращающие:
\begin{enumerate}[label*=\arabic*.]
	\item второй;
	\begin{code}
		\inputminted
		[
		frame=single,
		framerule=0.5pt,
		framesep=10pt,
		fontsize=\small,
		tabsize=4,
		linenos,
		numbersep=5pt,
		xleftmargin=10pt,
		]
		{text}
		{code/1.lisp}
	\end{code}

	\item третий;
	\begin{code}
		\inputminted
		[
		frame=single,
		framerule=0.5pt,
		framesep=10pt,
		fontsize=\small,
		tabsize=4,
		linenos,
		numbersep=5pt,
		xleftmargin=10pt,
		]
		{text}
		{code/2.lisp}
	\end{code}
	
	\item четвертый элементы заданного списка.
		\begin{code}
		\inputminted
		[
		frame=single,
		framerule=0.5pt,
		framesep=10pt,
		fontsize=\small,
		tabsize=4,
		linenos,
		numbersep=5pt,
		xleftmargin=10pt,
		]
		{text}
		{code/3.lisp}
	\end{code}
\end{enumerate}



\subsection{Задание \textnumero3}

Что будет в результате вычисления выражений?

\begin{enumerate}[label*=\arabic*.]
	\item  (CAADR $'$((blue cube) (red pyramid)))
	
	red
	\item  (CDAR $'$((abc) (def) (ghi)))
	
	nil
	
	\item (CADR $'$((abc) (def) (ghi)))
	
	(def)
	
	\item  (CADDR $'$((abc) (def) (ghi)))
	
	(ghi)

\end{enumerate}

\subsection{Задание \textnumero4}

Напишите результат вычисления выражений и объясните как он получен:
\begin{code}
	\inputminted
	[
	frame=single,
	framerule=0.5pt,
	framesep=10pt,
	fontsize=\small,
	tabsize=4,
	linenos,
	numbersep=5pt,
	xleftmargin=10pt,
	]
	{text}
	{code/4.lisp}
\end{code}

\subsection{Задание \textnumero5}

Написать лямбда-выражение и соответствующую функцию:
\begin{enumerate}[label*=\arabic*.]
	\item Написать функцию (f arl ar2 ar3 ar4), возвращающую список: ((arl ar2) (ar3 ar4)).
	\begin{code}
		\inputminted
		[
		frame=single,
		framerule=0.5pt,
		framesep=10pt,
		fontsize=\small,
		tabsize=4,
		linenos,
		numbersep=5pt,
		xleftmargin=10pt,
		]
		{text}
		{code/func1.lisp}
	\end{code}
	
	\item Написать функцию (f arl ar2), возвращающую ((arl) (ar2)).
	\begin{code}
		\inputminted
		[
		frame=single,
		framerule=0.5pt,
		framesep=10pt,
		fontsize=\small,
		tabsize=4,
		linenos,
		numbersep=5pt,
		xleftmargin=10pt,
		]
		{text}
		{code/func2.lisp}
	\end{code}
	
	\item Написать функцию (f arl), возвращающую (((arl))). 
	\begin{code}
		\inputminted
		[
		frame=single,
		framerule=0.5pt,
		framesep=10pt,
		fontsize=\small,
		tabsize=4,
		linenos,
		numbersep=5pt,
		xleftmargin=10pt,
		]
		{text}
		{code/func3.lisp}
	\end{code}
\end{enumerate} 

Результаты в виде списочных ячеек представлены на отдельном листе.
