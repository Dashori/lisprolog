\section{Практические задания}

\subsection{Задание \textnumero1}

Выполнено на отдельном листе.

\subsection{Задание \textnumero2}

Написать функцию, вычисляющую гипотенузу прямоугольного треугольника по заданным катетам и составить диаграмму ее вычисления.

\begin{code}
	\inputminted
	[
	frame=single,
	framerule=0.5pt,
	framesep=10pt,
	fontsize=\small,
	tabsize=4,
	linenos,
	numbersep=5pt,
	xleftmargin=10pt,
	]
	{text}
	{code/2.lisp}
\end{code}

Диаграмма вычислений выполнена на отдельном листе.


\subsection{Задание \textnumero3}

Каковы результаты вычисления следующих выражений? (объяснить возможную ошибку и варианты ее устранения)

\begin{enumerate}[label*=\arabic*.]
	\item (list 'a c)
	
	Ошибка, ибо C не самовычисляема, нужно добавить ', то есть 'c. Тогда ответ (A C ).
	
	\item  	(cons 'a (b c))
	
	Ошибка, ибо попытается вычислить (b c). Нужно добавить ' и получить '(b c). Тогда ответ (A B C).
	
	\item (cons 'a '(b c))
	
	(A B C)
	
	\item  (caddr (1 2 3 4 5))
	
	Ошибка, ибо (1 2 3 4 5) надо представить как список '(1 2 3 4 5). Тогда выведет 3 элемент, то есть 3. 
	
	\item (cons 'a 'b 'c)
	
	Ошибка, у cons только два аргумента. Если надо создать список из трех элементов a b c, то с помощью cons это можно сделать как в пункте 2.
	
	\item (list 'a (b c))
	
	Ошибка, список (b c) надо представить как '(b c), тогда ответ (A (B C)).
	
	\item (list a '(b c))
	
	Ошибка, a надо представить как '(a), тогда ответ (A (B C)).
		
	\item (list (+ 1 '(length '(1 2 3))))
	
	Ошибка,  ибо  '(length '(1 2 3)) не является числом. Убрать ' и ответ (4).

\end{enumerate}

\subsection{Задание \textnumero4}

Написать функцию longer\_then от двух списков-аргументов, которая возвращает Т, если первый аргумент имеет большую длину.

\begin{code}
	\inputminted
	[
	frame=single,
	framerule=0.5pt,
	framesep=10pt,
	fontsize=\small,
	tabsize=4,
	linenos,
	numbersep=5pt,
	xleftmargin=10pt,
	]
	{text}
	{code/4.lisp}
\end{code}

\subsection{Задание \textnumero5}

Каковы результаты вычисления следующих выражений:
\begin{enumerate}[label*=\arabic*.]
	\item (cons 3 (list 5 6))
	
	(3 5 6)

	\item (list 3 'from 9 'lives (- 9 3))
	
	(3 FROM 9 LIVES 6)
	
	\item (+ (length for 2 too)) (car '(21 22 23)))
	
	Ошибка
	
	\item (cdr '(cons is short for ans))
	
	(IS SHORT FOR ANS)
	
	\item (car (list one two))
	
	Ошибка
	
	\item (cons 3 '(list 5 6))
	
	(3 LIST 5 6)
	
	\item (car (list 'one 'two))

	ONE	
\end{enumerate} 

\subsection{Задание \textnumero6}

Дана функция (defun mystery (x) (list (second x) (first x))). Какие результаты вычисления следующих выражений?


\begin{enumerate}[label*=\arabic*.]
	\item (mystery (one two))
	
	Ошибка, необходимо добавить ' к списку.

	\item (mystery (last one two))
	
	Ошибка, необходимо добавить ' к списку.
	
	\item (mystery free)
	
	Ошибка, необходимо добавить '  и скобки к free.
	
	\item (mystery one 'two))
	
	Ошибка, ибо ожидается один элемент. 
	
\end{enumerate} 


\subsection{Задание \textnumero7}

 Написать функцию, которая переводит температуру в системе Фаренгейта температуру по Цельсию (defun f-to-c (temp)...).
 
Формулы: c = 5/9*(f-320); f= 9/5*c+32.0.

Как бы назывался роман Р.Брэдбери "+451 по Фаренгейту" в системе по Цельсию?

\begin{code}
	\inputminted
	[
	frame=single,
	framerule=0.5pt,
	framesep=10pt,
	fontsize=\small,
	tabsize=4,
	linenos,
	numbersep=5pt,
	xleftmargin=10pt,
	]
	{text}
	{code/7.lisp}
\end{code}

\subsection{Задание \textnumero8}

Что получится при вычисления каждого из выражений?

\begin{enumerate}[label*=\arabic*.]
	\item (list 'cons t NIL)
	
	(CONS T NIL)
	
	\item (eval (eval (list 'cons t NIL)))
	
 	Ошибка.
	
	\item (apply \#cons "(t NIL))
	
	Это будет  результат cons '  '(t nil). Получается (' T NIL).
	
	\item (list 'eval NIL)
	
	(EVAL NIL)

	\item (eval (list 'cons t NIL))
	
	(T)
	
	\item (eval NIL)
	
	(NIL)
	
	\item  (eval (list 'eval NIL))
	
	 (NIL)
	
\end{enumerate} 


