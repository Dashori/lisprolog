\section{Практические задания}

\subsection{Задание \textnumero1}

Пусть (setf lst1 '( a b c))

(setf lst2 '( d e))

Каковы результаты вычисления следующих выражений?

\begin{enumerate}[label*=\arabic*.]
	\item (cons lstl lst2) 
	
	((A B C) D E)
	
	\item (list lst1 lst2) 
	
	((A B C) ( D E))
	
	\item (append lst1 lst2)
	
	(A B C D E)
	
\end{enumerate} 


\subsection{Задание \textnumero2}

Каковы результаты вычисления следующих выражений, и почему?

\begin{enumerate} 

\item (reverse '(a b c)) 

(C B A) -- обычный разворот списка

\item (reverse '(a b (c (d)))) 

((C (D))  B A)  -- так как reverse работает относительно элементов списка верхнего уровня

\item (reverse '(a)) 

(A)
\item (reverse ())

NIL

\item (reverse '((a b c)))

((A B C)) -- так как reverse работает относительно элементов списка верхнего уровня, а (A B C) единственный элемент, то он останется на своем месте

\item (last '(a b c))

(С) -- так как является последним элементом

\item (last '(a))

(A) -- так как является единственным, то есть и первым и последним элементом

\item (last '((a b c)))

((A B C)) -- так как last работает относительно элементов списка верхнего уровня, а (A B C) единственный элемент, то есть и первый и последний

\item (last '(a b (c))) 

((С)) -- последний элемент списка верхнего уровня

\item (last ())

(NIL)

\end{enumerate} 

\subsection{Задание \textnumero3}

Написать, по крайней мере, два варианта функции, которая возвращает последний элемент своего списка-аргумента.

 \begin{code}
 	\inputminted
 	[
 	frame=single,
 	framerule=0.5pt,
 	framesep=10pt,
 	fontsize=\small,
 	tabsize=4,
 	linenos,
 	numbersep=5pt,
 	xleftmargin=10pt,
 	]
 	{text}
 	{code/3.lisp}
 \end{code}
 

\subsection{Задание \textnumero4}

Написать, по крайней мере, два варианта функции, которая возвращает свой список аргумент без последнего элемента.

\begin{code}
	\inputminted
	[
	frame=single,
	framerule=0.5pt,
	framesep=10pt,
	fontsize=\small,
	tabsize=4,
	linenos,
	numbersep=5pt,
	xleftmargin=10pt,
	]
	{text}
	{code/4.lisp}
\end{code}

\subsection{Задание \textnumero5}

Напишите функцию swap-first-last, которая переставляет в списке- аргументе первый и последний элементы.

\begin{code}
	\inputminted
	[
	frame=single,
	framerule=0.5pt,
	framesep=10pt,
	fontsize=\small,
	tabsize=4,
	linenos,
	numbersep=5pt,
	xleftmargin=10pt,
	]
	{text}
	{code/5.lisp}
\end{code} 

\subsection{Задание \textnumero6}

 Написать простой вариант игры в кости, в котором бросаются две правильные кости. Если сумма выпавших очков равна 7 или 11 — выигрыш, если выпало (1,1) или (6,6) — игрок имеет право снова бросить кости, во всех остальных случаях ход переходит ко второму игроку, но запоминается сумма выпавших очков. Если второй игрок не выигрывает абсолютно, то выигрывает тот игрок, у которого больше очков. Результат игры и значения выпавших костей выводить на экран с помощью функции print.

\begin{code}
	\inputminted
	[
	frame=single,
	framerule=0.5pt,
	framesep=10pt,
	fontsize=\small,
	tabsize=4,
	linenos,
	numbersep=5pt,
	xleftmargin=10pt,
	]
	{text}
	{code/6.lisp}
\end{code}

\subsection{Задание \textnumero7}

Написать функцию, которая по своему списку-аргументу lst определяет является ли он палиндромом (то есть равны ли lst и (reverse lst)).

\begin{code}
	\inputminted
	[
	frame=single,
	framerule=0.5pt,
	framesep=10pt,
	fontsize=\small,
	tabsize=4,
	linenos,
	numbersep=5pt,
	xleftmargin=10pt,
	]
	{text}
	{code/7.lisp}
\end{code}


\subsection{Задание \textnumero8}

Напишите свои необходимые функции, которые обрабатывают таблицу из 4-х точечных пар: (страна . столица), и возвращают по стране -- столицу, а по столице -- страну.

\begin{code}
	\inputminted
	[
	frame=single,
	framerule=0.5pt,
	framesep=10pt,
	fontsize=\small,
	tabsize=4,
	linenos,
	numbersep=5pt,
	xleftmargin=10pt,
	]
	{text}
	{code/8.lisp}
\end{code}

\subsection{Задание \textnumero9}

Напишите функцию, которая умножает на заданное число-аргумент первый числовой элемент списка из заданного 3-х элементного списка-аргумента, когда все элементы списка -- числа/элементы списка -- любые объекты.

\begin{code}
	\inputminted
	[
	frame=single,
	framerule=0.5pt,
	framesep=10pt,
	fontsize=\small,
	tabsize=4,
	linenos,
	numbersep=5pt,
	xleftmargin=10pt,
	]
	{text}
	{code/9.lisp}
\end{code}
