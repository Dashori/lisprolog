\section{Практические задания}

\subsection{Задание \textnumero1}

Написать функцию, которая принимает целое число и возвращает первое четное число, не меньшее аргумента.

\begin{code}
	\inputminted
	[
	frame=single,
	framerule=0.5pt,
	framesep=10pt,
	fontsize=\small,
	tabsize=4,
	linenos,
	numbersep=5pt,
	xleftmargin=10pt,
	]
	{text}
	{code/1.lisp}
\end{code}


\subsection{Задание \textnumero2}


Написать функцию, которая принимает число и возвращает число того же знака, но с модулем на 1 больше модуля аргумента.

\begin{code}
	\inputminted
	[
	frame=single,
	framerule=0.5pt,
	framesep=10pt,
	fontsize=\small,
	tabsize=4,
	linenos,
	numbersep=5pt,
	xleftmargin=10pt,
	]
	{text}
	{code/2.lisp}
\end{code}

\subsection{Задание \textnumero3}

 Написать функцию, которая принимает два числа и возвращает список из этих чисел, расположенный по возрастанию.
 
 \begin{code}
 	\inputminted
 	[
 	frame=single,
 	framerule=0.5pt,
 	framesep=10pt,
 	fontsize=\small,
 	tabsize=4,
 	linenos,
 	numbersep=5pt,
 	xleftmargin=10pt,
 	]
 	{text}
 	{code/3.lisp}
 \end{code}
 

\subsection{Задание \textnumero4}

Написать функцию, которая принимает три числа и возвращает Т только тогда, когда первое число расположено между вторым и третьим.

\begin{code}
	\inputminted
	[
	frame=single,
	framerule=0.5pt,
	framesep=10pt,
	fontsize=\small,
	tabsize=4,
	linenos,
	numbersep=5pt,
	xleftmargin=10pt,
	]
	{text}
	{code/4.lisp}
\end{code}

\subsection{Задание \textnumero5}

Каковы результаты вычисления следующих выражений:

\begin{enumerate}[label*=\arabic*.]
	\item (and 'fee 'fie 'foe)
	
	FOE
	
	\item (or 'fee 'fie 'foe)
	
	FEE
	
	\item (and nil 'fie 'foe)	
	
	NIL
		
	\item (or nil 'fie 'foe)
	
	FIE
	
	\item (and (equal 'abc 'abc) 'yes)
	
	YES
	
	\item (or (equal 'abc 'abc) 'yes)
	
	T
	
\end{enumerate} 

\subsection{Задание \textnumero6}

Написать предикат, который принимает два числа-аргумента и возвращает Т, если первое число не меньше второго.

\begin{code}
	\inputminted
	[
	frame=single,
	framerule=0.5pt,
	framesep=10pt,
	fontsize=\small,
	tabsize=4,
	linenos,
	numbersep=5pt,
	xleftmargin=10pt,
	]
	{text}
	{code/6.lisp}
\end{code}

\subsection{Задание \textnumero7}

Какой из следующих двух вариантов предиката ошибочен и почему?

(defun pred1 (x)
(and (numberp x) (plusp x))) 

(defun pred2 (x)
(and (plusp x)(numberp x)))

Ошибочен второй, так как вначале проверяется <<знак>>,  а только после этого проверяется является ли x числом. Предикат plusp определен только для чисел, а значит не для чисел интерпретатор выдаст ошибку. 

\subsection{Задание \textnumero8}

Решить задачу 4, используя для ее решения конструкции: только IF, только COND, только AND/OR.
Только IF  реализовано в задаче 4.

Только COND:
\begin{code}
	\inputminted
	[
	frame=single,
	framerule=0.5pt,
	framesep=10pt,
	fontsize=\small,
	tabsize=4,
	linenos,
	numbersep=5pt,
	xleftmargin=10pt,
	]
	{text}
	{code/8-cond.lisp}
\end{code}

Только AND/OR:
\begin{code}
	\inputminted
	[
	frame=single,
	framerule=0.5pt,
	framesep=10pt,
	fontsize=\small,
	tabsize=4,
	linenos,
	numbersep=5pt,
	xleftmargin=10pt,
	]
	{text}
	{code/8-and.lisp}
\end{code}


\subsection{Задание \textnumero9}

Переписать функцию how-alike, приведенную в лекции и использующую COND, используя только конструкции IF, AND/OR.

\begin{code}
	\inputminted
	[
	frame=single,
	framerule=0.5pt,
	framesep=10pt,
	fontsize=\small,
	tabsize=4,
	linenos,
	numbersep=5pt,
	xleftmargin=10pt,
	]
	{text}
	{code/how-alik-cond.lisp}
\end{code}

Только IF:
\begin{code}
	\inputminted
	[
	frame=single,
	framerule=0.5pt,
	framesep=10pt,
	fontsize=\small,
	tabsize=4,
	linenos,
	numbersep=5pt,
	xleftmargin=10pt,
	]
	{text}
	{code/how-alik-if.lisp}
\end{code}

Только AND/OR:
\begin{code}
	\inputminted
	[
	frame=single,
	framerule=0.5pt,
	framesep=10pt,
	fontsize=\small,
	tabsize=4,
	linenos,
	numbersep=5pt,
	xleftmargin=10pt,
	]
	{text}
	{code/how-alik-and.lisp}
\end{code}

