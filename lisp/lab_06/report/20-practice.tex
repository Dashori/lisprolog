\section{Практические задания}
Используя функционалы:

\subsection{Задание \textnumero1}

 Написать хвостовую рекурсивную функцию my-reverse, которая развернет верхний уровень своего списка-аргумента lst.
 
 \begin{code}
	\inputminted
	[
	frame=single,
	framerule=0.5pt,
	framesep=10pt,
	fontsize=\small,
	tabsize=4,
	linenos,
	numbersep=5pt,
	xleftmargin=10pt,
	]
	{text}
	{code/1.lisp}
\end{code}

\subsection{Задание \textnumero2}

Написать функцию, которая возвращает первый элемент списка-аргумента, который сам является непустым списком.

 \begin{code}
	\inputminted
	[
	frame=single,
	framerule=0.5pt,
	framesep=10pt,
	fontsize=\small,
	tabsize=4,
	linenos,
	numbersep=5pt,
	xleftmargin=10pt,
	]
	{text}
	{code/2.lisp}
\end{code}

\subsection{Задание \textnumero3}

Напишите рекурсивную функцию, которая умножает на заданное число-аргумент все числа из заданного списка-аргумента, когда: все элементы списка -- числа,  элементы списка -- любые объекты

 \begin{code}
 	\inputminted
 	[
 	frame=single,
 	framerule=0.5pt,
 	framesep=10pt,
 	fontsize=\small,
 	tabsize=4,
 	linenos,
 	numbersep=5pt,
 	xleftmargin=10pt,
 	]
 	{text}
 	{code/3.lisp}
 \end{code}
 

\subsection{Задание \textnumero4}

Напишите функцию, select-between, которая из списка-аргумента, содержащего только числа, выбирает только те, которые расположены между двумя указанными границами- аргументами и возвращает их в виде списка

\begin{code}
	\inputminted
	[
	frame=single,
	framerule=0.5pt,
	framesep=10pt,
	fontsize=\small,
	tabsize=4,
	linenos,
	numbersep=5pt,
	xleftmargin=10pt,
	]
	{text}
	{code/4.lisp}
\end{code}

\subsection{Задание \textnumero5}

Написать рекурсивную версию (с именем rec-add) вычисления суммы чисел заданного списка: одноуровнего смешанного, структурированного.

\begin{code}
	\inputminted
	[
	frame=single,
	framerule=0.5pt,
	framesep=10pt,
	fontsize=\small,
	tabsize=4,
	linenos,
	numbersep=5pt,
	xleftmargin=10pt,
	]
	{text}
	{code/5.lisp}
\end{code} 
