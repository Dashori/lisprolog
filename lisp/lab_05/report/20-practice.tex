\section{Практические задания}
Используя функционалы:

\subsection{Задание \textnumero1}

Напишите функцию, которая уменьшает на 10 все числа из списка-аргумента этой
функции, проходя по верхнему уровню списковых ячеек. ( * Список смешанный структурированный)

 \begin{code}
	\inputminted
	[
	frame=single,
	framerule=0.5pt,
	framesep=10pt,
	fontsize=\small,
	tabsize=4,
	linenos,
	numbersep=5pt,
	xleftmargin=10pt,
	]
	{text}
	{code/1.lisp}
\end{code}

\subsection{Задание \textnumero2}

Написать функцию которая получает как аргумент список чисел, а возвращает список квадратов этих чисел в том же порядке.
 \begin{code}
	\inputminted
	[
	frame=single,
	framerule=0.5pt,
	framesep=10pt,
	fontsize=\small,
	tabsize=4,
	linenos,
	numbersep=5pt,
	xleftmargin=10pt,
	]
	{text}
	{code/2.lisp}
\end{code}

\pagebreak

\subsection{Задание \textnumero3}

Напишите функцию, которая умножает на заданное число-аргумент все числа из заданного списка-аргумента, когда: все элементы списка -- числа, элементы списка -- любые объекты.
 \begin{code}
 	\inputminted
 	[
 	frame=single,
 	framerule=0.5pt,
 	framesep=10pt,
 	fontsize=\small,
 	tabsize=4,
 	linenos,
 	numbersep=5pt,
 	xleftmargin=10pt,
 	]
 	{text}
 	{code/3.lisp}
 \end{code}
 

\subsection{Задание \textnumero4}

Написать функцию, которая по своему списку-аргументу lst определяет является ли он палиндромом (то есть равны ли lst и (reverse lst)), для одноуровнего смешанного списка.

\begin{code}
	\inputminted
	[
	frame=single,
	framerule=0.5pt,
	framesep=10pt,
	fontsize=\small,
	tabsize=4,
	linenos,
	numbersep=5pt,
	xleftmargin=10pt,
	]
	{text}
	{code/4.lisp}
\end{code}

\subsection{Задание \textnumero5}

Используя функционалы, написать предикат set-equal, который возвращает t, если два его множества-аргумента (одноуровневые списки) содержат одни и те же элементы, порядок которых не имеет значения.
\begin{code}
	\inputminted
	[
	frame=single,
	framerule=0.5pt,
	framesep=10pt,
	fontsize=\small,
	tabsize=4,
	linenos,
	numbersep=5pt,
	xleftmargin=10pt,
	]
	{text}
	{code/5.lisp}
\end{code} 

\subsection{Задание \textnumero6}

Напишите функцию, select-between, которая из списка-аргумента, содержащего только числа, выбирает только те, которые расположены между двумя указанными числами -- границами-аргументами и возвращает их в виде списка (упорядоченного по возрастанию (+ 2 балла)).

\begin{code}
	\inputminted
	[
	frame=single,
	framerule=0.5pt,
	framesep=10pt,
	fontsize=\small,
	tabsize=4,
	linenos,
	numbersep=5pt,
	xleftmargin=10pt,
	]
	{text}
	{code/6.lisp}
\end{code}

\subsection{Задание \textnumero7}

Написать функцию, вычисляющую декартово произведение двух своих списков- аргументов. ( Напомним, что А х В это множество всевозможных пар (a b), где а принадлежит А, принадлежит В.)

\begin{code}
	\inputminted
	[
	frame=single,
	framerule=0.5pt,
	framesep=10pt,
	fontsize=\small,
	tabsize=4,
	linenos,
	numbersep=5pt,
	xleftmargin=10pt,
	]
	{text}
	{code/7.lisp}
\end{code}


\subsection{Задание \textnumero8}

Почему так реализовано reduce, в чем причина?

$(reduce \#'+ ()) -> 0$

Поведение в данном примере обусловлено работой функции +. Эта функция -- функционал, который при 0 количестве аргументов возвращает значение 0. Если подать на вход reduce функцию, которая не может обработать 0 аргументов, то вызов reduce с пустым списком в качестве второго аргумента вернет ошибку. При этом, если подано более одного аргумента, то reduce выполняет следующие действия:

\begin{enumerate}[label=---]
	\item сохраняет первый элемент списка в область памяти (acc);
	\item  для всех остальных элементов списка выполняет переданную в качестве первого аргумента функцию, подавая на вход 2 аргумента (acc и очередной элемент списка) и сохраняя результат в acc.
\end{enumerate}

$(reduce \#'* ()) -> 1$

Для умножения ситуация аналогичная.

\subsection{Задание \textnumero9}

Пусть list-of-list список, состоящий из списков. Написать функцию, которая вычисляет сумму длин всех элементов list-of-list (количество атомов), т.е. например для аргумента ((1 2) (3 4)) -> 4.

\begin{code}
	\inputminted
	[
	frame=single,
	framerule=0.5pt,
	framesep=10pt,
	fontsize=\small,
	tabsize=4,
	linenos,
	numbersep=5pt,
	xleftmargin=10pt,
	]
	{text}
	{code/9.lisp}
\end{code}
